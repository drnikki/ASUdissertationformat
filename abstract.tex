\begin{abstract}string that can uniquely identify it. Primary keys are essential for linking data spread across database tables,
and for looking up and retrieving data from specific records. Yet for an identifier that seems so straightforward
and uncontroversial, we find myriad ways that this unassuming bit of infrastructure has an outsized influence
in human services work. Through case studies of the organizational networks of two nonprofit human services
organizations, we find that different stakeholders use variants of identifiers to support work practices that are
far more complex and social than the linking of tables or the lookup of data. Yet we also find that the low-level
technical properties of the primary key are often coercive, forcing end-users to work on the infrastructure’s
terms—influencing the nature and order of the work, creating new forms of work, and influencing the tenor
of the relationships among stakeholders\end{abstract}